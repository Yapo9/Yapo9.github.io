\documentclass[11pt,a4paper,sans]{moderncv}

% Author: Kaihang JI
% Github Repo: https://github.com/mimicji/Bilingual-Resume-Template
% Requirement: XeLaTeX

% 使用方法:
%   ***必须使用XeLaTeX编译***
%   撰写简历时:在\CN{}内填写中文内容,后其紧跟的\EN{}内填写英文内容
%   导出时: 选择[Chinese]为中文版,选择[English]为英文版
%\usepackage[Chinese]{languageSelection} % 导出中文版
\usepackage[English]{languageSelection} % 导出英文版

\moderncvstyle{banking}
\moderncvcolor{black}
\nopagenumbers{}
\usepackage[utf8]{inputenc}
\usepackage{ragged2e}
\usepackage[scale=0.915]{geometry}
\usepackage{import}
\usepackage{multicol}
\usepackage{enumitem}
\usepackage{amssymb}
\usepackage{fontawesome5}
\usepackage{zh_CN-Adobefonts_external}
\usepackage{xeCJK}

\newcommand*{\customcventry}[7][.13em]{
\begin{tabular}{@{}l}
{\bfseries #4} \
{\itshape #3}
\end{tabular}
\hfill
\begin{tabular}{l@{}}
{\bfseries #5} \
{\itshape #2}
\end{tabular}
\ifx&#7&%
\else{\
\begin{minipage}{\maincolumnwidth}%
\small#7%
\end{minipage}}\fi%
\par\addvspace{#1}}

%**************************************************
%*                 从这里开始                       *
%**************************************************

% 你的双语姓名
\name{付亚鹏}{Ya-Peng(Evan)Fu}

\begin{document}
\makecvtitle
\vspace*{-10mm}

% 个人title
\begin{center}
\vspace*{-5mm}
\CN{
    \textbf{经济学博士在读}
}
\EN{
    \textbf{Ph.D. student in Economics}
}
\end{center}


% 联系方式
\begin{center}
\begin{tabular}{ c c c }
\faPhone\enspace +86 15611577833 & \enspace
\faWeixin\enspace {yapeng\_f} & \enspace
\faHome\enspace 
\CN{北京,中国} \EN{Beijing, CN} \\
\faEnvelope 1:\enspace
\color{blue} 
\href{mailto:yapengf@outlook.com}
{yapengf@outlook.com} & \enspace
\faEnvelope 2:\enspace \color{blue}  \href{mailto:fuyapeng@ucass.edu.cn}
{fuyapeng@ucass.edu.cn} & \enspace \faLink
\enspace \color{blue} 
\href{https://yapengf.com}
{yapengf.com}
\end{tabular}
\end{center}

% 以上项目如需增减,则其它可选图标为:
% \faWhatsapp         % Whatsapp
% \faTwitter          % Twitter
% \faLinkedin         % Linkedin
% \faLine             % Line
% \faFacebookSquare   % Facebook
% \faInstagram        % Instagram
% \faZhihu            % 知乎
% 更多图标见: https://mirrors.ibiblio.org/CTAN/fonts/fontawesome5/doc/fontawesome5.pdf

\begin{center}
    \CN{\underline{更新于: \today} } \EN{\underline{Latest Update: \today}}
\end{center}

% 教育经历
\CN{\section{\faGraduationCap \enspace 教育经历}}
\EN{\section{\faGraduationCap \enspace Education}}

\CN{
\customcventry{09/2024 - 06/2029(预计)}
{{中国社会科学院大学}}
{经济学博士(在读)}{}{}{}
}
\EN{
\customcventry{09/2024 - 06/2029(expected)}
{{University of Chinese Academy of Social Sciences}}
{Ph.D.(Enrolled) in Economics}{}{}{}
}

\CN{
\customcventry{09/2020 - 06/2024}
{{中国社会科学院大学}}
{经济学学士}{}{}{}
}
\EN{
\customcventry{09/2020 - 06/2024}
{{University of Chinese Academy of Social Sciences}}
{B.S. in Economics}{}{}{}
}

% 研究兴趣
\CN{\section{\faBookOpen \enspace 研究兴趣}}
\EN{\section{\faBookOpen \enspace Research Interests}}
\CN{我的研究兴趣主要集中在\textbf{社会保障}、\textbf{微观经济理论}、\textbf{卫生经济学}和\textbf{公共经济学}。最近,我正在从事一些与公共政策评估、医疗保险研究、财政和税收政策等有关的研究。}
\EN{My research interests focus on \textbf{social security}, \textbf{microeconomic theory}, \textbf{health economics} and \textbf{public economics}. Recently, I have been working on some research related to public policy evaluation, mathematical methods of economics, medical insurance research, fiscal and tax policies, etc.}

% 工作论文
\CN{\section{\faBook \enspace 工作论文}}
\EN{\section{\faBook \enspace Working Papers}}
{\begin{itemize}[label=\textbullet]
\item 
\CN{
    \textbf{医保支付改革、道德风险与医疗控费——来自Bunching-DID的证据}(2024,合作者:丛正龙、倪晨旭、王震)
}
\EN{
    \textbf{Medical Insurance Payment Reform, Moral Hazard, and Medical Cost Control: Evidence from Bunching-DID} (2024, with Zhenglong Cong, Chenxu Ni and Zhen Wang) 
}

\item 
\CN{
    \textbf{土地财政与政府隐性债务:来自聚束估计的证据}(2023,合作者:丛正龙)
}
\EN{
    \textbf{Land Finance and Government Implicit Debt: Evidence from Bunching} (2023, with Zhenglong Cong)
}

\end{itemize}} 

% 其他项目
\CN{\section{\faBook \enspace 其他论文}}
\EN{\section{\faBook \enspace Other Projects}}
{\begin{itemize}[label=\textbullet]
\item 
\CN{
    \textbf{越畔之思:土地财政转型中的政府市场互动研究}(2023国家级大学生创新训练计划项目,合作者:李芃辰,周理郡)
}
\EN{
    \textbf{Egg in beer: Research on Government-Market Interaction in the Transformation of Land Finance in China} (2023 College Students Innovative Training Program, with Pengchen Li and Lijun Zhou)
}

\item 
\CN{
    \textbf{财政激励、市场整合与新型城镇化——基于“撤县设区”的多时点双重差分研究}(2022《计量经济学》课程论文,合作者:韩登清)
}
\EN{
    \textbf{Fiscal Incentives, Market Integration, and New-Type Urbanization: Research on "County-to-District Conversion" based on Staggered DID} (2022 Term Paper on Econometrics, with Dengqing Han)
}
\item 
\CN{
    \textbf{舆论情绪、关联报道与股价波动}(2022中国社会科学院大学大学生创新训练计划项目,合作者:韩登清 丛正龙)
}
\EN{
    \textbf{Public Sentiment, Associated Reporting, and Stock Price Volatility} (2022 UCASS Students Innovative Training Program, with Zhenglong Cong and Dengqing Han)
}
\item 
\CN{
    \textbf{转移支付、市场整合与基本公共服务均等化}(2022《中国经济专题》课程论文,合作者:韩登清)
}
\EN{
    \textbf{Intergovernmental Transfers, Market Integration, and Equitization of Basic Public Services} (2022 Term Paper on China's Economy, with Dengqing Han)
}

\end{itemize}} 

% 获奖荣誉
\CN{\section{\faAward \enspace 获奖荣誉}}
\EN{\section{\faAward \enspace Awards}}

\CN{07/2024: 北京市高校优秀毕业论文(本科)\enspace \hfill{北京}}
\EN{07/2024: the Beijing Outstanding Graduation Dissertation(B.S.) \enspace \hfill{Beijing}}

\CN{05/2024: 北京市高校优秀毕业生(本科)\enspace \hfill{北京}}
\EN{05/2024: Outstanding Graduates of Beijing(B.S.) \enspace \hfill{Beijing}}

\CN{12/2023: 国家励志奖学金(两次)\enspace \hfill{北京}}
\EN{12/2023: National Inspiration Scholarship(Twice)  \enspace \hfill{Beijing}}

\CN{09/2023: 第十八届“挑战杯”大学生全国课外学术科技作品竞赛三等奖\enspace \hfill{贵州}}
\EN{09/2023: Third Prize of the 18th "Challenge Cup" NCSCASTWC  \enspace \hfill{Guizhou}}

\CN{07/2023: 第三届全国大学生发展经济学论文大赛一等奖\enspace \hfill{北京}}
\EN{07/2023: First Prize of the 3rd National Essay Competition on Development Economics \enspace \hfill{Beijing}}

\CN{07/2023: 第三届“经英杯”全国高校经济论坛一等奖\enspace \hfill{北京}}
\EN{07/2023: First Prize of the 3rd “Jingying Cup” National Economic Forum in RUC \enspace \hfill{Beijing}}

\CN{06/2023: 首届全国大学生公共经济与政策论文大赛二等奖\enspace \hfill{厦门}}
\EN{06/2023: Second Prize of the First National Public Economy and Policy Paper Competition  \enspace \hfill{Xiamen}}

\CN{05/2023: 第十二届新时代中国青年经济论坛优秀学术代表\enspace \hfill{北京}}
\EN{05/2023: Outstanding Academic Delegate of the 12th China Youth Economic Forum in PKU \enspace \hfill{Beijing}}

\CN{04/2023: 中国社会科学院大学优秀共青团员\enspace \hfill{北京}}
\EN{04/2023: Excellent League Member in UCASS  \enspace \hfill{Beijing}}

\CN{12/2022: 中国社会科学院大学优秀学生干部\enspace \hfill{北京}}
\EN{12/2022: Excellent League Cadres in UCASS  \enspace \hfill{Beijing}}

\CN{12/2021: 第十三届全国大学生数学竞赛二等奖\enspace \hfill{北京}}
\EN{12/2021: Second Prize of the 13th Chinese Mathematics Competitions  \enspace \hfill{Beijing}}


% 学术活动
\CN{\section{\faCalendar* \enspace 学术活动}}
\EN{\section{\faCalendar* \enspace Academic Activities}}

\CN{07/2024: 第八届中国劳动经济学者论坛年会\enspace \hfill{上海}}
\EN{07/2024: The 8th Annual Meeting of the China Labor Economics Scholars Forum \enspace \hfill{Shanghai}}

\CN{06/2024: 第二届香樟社会保障论坛\enspace \hfill{上海}}
\EN{06/2024: The 2nd Camphor Forum for Social Security  \enspace \hfill{Shanghai}}

\CN{05/2023: 第十二届新时代中国青年经济论坛\enspace \hfill{北京}}
\EN{05/2023: The 12th China Youth Economic Forum in SOE of PKU \enspace \hfill{Beijing}}

\CN{04/2023: 第五届国家发展青年论坛\enspace \hfill{北京}}
\EN{04/2023: The 5th National Development Youth Forum in NSD of PKU\enspace \hfill{Beijing}}


\end{document}